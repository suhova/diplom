  
\chapter*{Введение} % * не проставляет номер
\addcontentsline{toc}{chapter}{Введение} % вносим в содержание

%Актуальность
Составление расписания экзаменационных сессий является сложным и кропотливым процессом, который требует автоматизации. Создание программы, которая поможет сотрудникам университета с предварительным распределением экзаменов по датам, времени и аудиториям с учётом пожеланий преподавателей и университетского расписания, сможет значительно ускорить процесс составления расписания. Диспетчер не всегда имеет возможность учесть предпочтения преподавателей.
Сложность составления расписания сессии заключается в переборе вариантов расстановки аттестаций и сверкой их с семестровым расписанием во избежание накладок с занятостью кабинетов и преподавателей. 

Таким образом, актуальность этой работы обуславливается необходимостью
\begin{itemize}
	\item формализовать сбор требований профессорско-преподавательского состава к проведению сессии;
	\item учитывать утверждённое расписание университета, опубликованное на сайте <<ruz.spbstu.ru>>;
	\item упростить процесс генерации предварительного расписания путём его автоматизации.
\end{itemize}

\textbf{Целью данной работы} является исследование алгоритмов для составления расписания сессии и проектирование сервиса составления предварительного расписания сессии СПбПУ.
Для достижения этой цели, в необходимо будет решить следующие \textbf{задачи}:

\begin{itemize}
	\item Провести анализ существующих систем составления расписания.
	\item Рассмотреть задачу составления расписания сессии с алгоритмической точки зрения, исследовать алгоритмы для составления расписания сессии и оптимизации этого процесса.
	\item Написать сервис для получения данных о расписании с использованием API <<ruz.spbstu.ru>>. % Это название программы
	\item Реализовать алгоритм составления предварительного расписания сессии. 
	\item Протестировать систему составления расписания.
\end{itemize}

%https://dep.spbstu.ru/userfiles/files/Polozhenie-o-GIA.pdf стр 28!!!

\textit{Практическая значимость работы} состоит в автоматизации труда диспетчеров СПбПУ по предварительной компоновке расписания для его последующего размещения на сайте.

%\textbf{НАПИСАТЬ СТРУКТУРУ ВКР как в шаблоне.}

\textit{Степень разработанности проблемы.} Проблема является NP-полной и является широко известной, однако технические особенности реализации для конкретной организации ранее рассмотрены не были (работа с API сайта расписания СПбПУ для учета существующего расписания).


В главе~\ref{ch1} приведен подробный обзор существующих программных систем по составлению расписания. Глава~\ref{ch2} посвящена обзору алгоритмов, которые можно применять для составления расписания сессии и оптимизации этого процесса. В главе~\ref{ch3} речь идёт о реализации разработанной системы, а в главе~\ref{ch4} описываются методы и результаты её тестирования и апробации.
