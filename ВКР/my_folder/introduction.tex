\chapter*{Введение} % * не проставляет номер
\addcontentsline{toc}{chapter}{Введение} % вносим в содержание

Составление расписания экзаменационных сессий является сложным и кропотливым процессом, который требует автоматизации. Создание программы, которая поможет сотрудникам университета с распределением экзаменов по датам, времени и аудиториям с учётом пожеланий преподавателей и университетского расписания, сможет значительно ускорить процесс составления расписания.
Сложность составления расписания сессии заключается в переборе вариантов расстановки аттестаций и сверкой их с семестровым расписанием во избежание накладок с занятостью кабинетов и преподавателей. 

Таким образом, актуальность этой работы обуславливается необходимостью
\begin{itemize}
	\item формализовать сбор требований профессорско-преподавательского состава к проведению сессии;
	\item учитывать утверждённое расписание университета, опубликованное на сайте ruz.spbstu.ru;
	\item упростить процесс генерации предварительного расписания путём его автоматизации.
\end{itemize}

\textbf{Целью данной работы} является исследование алгоритмов для составления расписания сессии и проектирование сервиса составления предварительного расписания сессии СПбПУ.
Для достижения этой цели, в необходимо будет решить следующие \textbf{задачи}:

\begin{itemize}
	\item Провести анализ существующих систем составления расписания.
	\item Рассмотреть задачу составления расписания сессии с алгоритмической точки зрения, исследовать алгоритмы для составления расписания сессии и оптимизации этого процесса.
	\item Написать сервис для получения данных о расписании с использованием API ruz.spbstu.ru
	\item Реализовать алгоритм составления предварительного расписания сессии. 
\end{itemize}
