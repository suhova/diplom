\begin{comment}
\chapter*{Введение} % * не проставляет номер
\addcontentsline{toc}{chapter}{Введение} % вносим в содержание

Составление расписания экзаменационных сессий сейчас является сложным и кропотливым процессом, который требует автоматизации. Создание программы, которая поможет сотрудникам университета с распределением экзаменов по датам, времени и аудиториям с учётом пожеланий преподавателей и университетского расписания, сможет значительно ускорить процесс составления расписания, а также устранит противоречия связанные с неустановленными формами сбора сведений. 

Сложность составления расписания сессии заключается в ручном переборе вариантов расстановки аттестаций и сверкой их с семестровым расписанием во избежание накладок с занятостью кабинетов и преподавателей. 

\textbf{Целью данной работы} является проектирование и реализация системы составления предварительного расписания сессии СПбПУ.
Для достижения этой цели, необходимо решить следующие \textbf{задачи}:

\begin{itemize}
	\item Провести анализ существующих систем составления расписания.
	\item Рассмотреть задачау составления расписания сессии с алгоритмической точки зрения, исследовать алгоритмы для составления расписания сессии и оптимизации этого процесса.
	\item Спроектировать систему составления предварительного расписания сессии СПбПУ.
	\item Разработать формы сбора сведений с преподавателей.
	\item Написать сервис для получения данных о расписании с использованием API ruz.spbstu.ru.
	\item Разработать сервис, находящий возможные варианты расписания сессиии на основе полученных сведений.
\end{itemize}
\end{comment}
\chapter*{Введение} % * не проставляет номер
\addcontentsline{toc}{chapter}{Введение} % вносим в содержание

%PA: во все временя является сложным...
Составление расписания экзаменационных сессий является сложным и кропотливым процессом, который требует автоматизации. Создание программы, которая поможет сотрудникам университета с распределением экзаменов по датам, времени и аудиториям с учётом пожеланий преподавателей и университетского расписания, сможет значительно ускорить процесс составления расписания. %, а также устранит противоречия связанные с неустановленными формами сбора сведений. %%PA: непонятно, о чем идет речь

%ПА: не обязательно в ручном
Сложность составления распсиания сессии заключается в переборе вариантов расстановки аттестаций и сверкой их с семестровым расписанием во избежание накладок с занятостью кабинетов и преподавателей. 

\textbf{СТРУКТУРИРУЕМ ВМЕСТЕ С АКТУАЛЬНОСТЬЮ, ОБЪЕКТОМ, ПРЕДМЕТОМ ИССЛЕДОВАНИЯ И ОПИСАНИЕМ ПОГЛАВНОЙ СТРУКТУРЫ ВКР, КАК МИНИМУМ (ИЗ ТОГО, ЧТО ЗАКОММЕНТИРОВАНО).}

\textbf{Целью данной работы} является исследование алгоритмов для составления расписания сессии и проектирование сервиса составления предварительного расписания сессии СПбПУ.
Для достижения этой цели, в необходимо будет решить следующие \textbf{задачи}:

\begin{itemize}
	\item Провести анализ существующих систем составления расписания.
	\item Рассмотреть задачу составления расписания сессии с алгоритмической точки зрения, исследовать алгоритмы для составления расписания сессии и оптимизации этого процесса.
	\item Написать сервис для получения данных о расписании с использованием API ruz.spbstu.ru
	\item Реализовать алгоритм составления предвариательного расписания сессии. % программно.
\end{itemize}
