\chapter{Многопоточная вариация алгоритма DFS}\label{appendix-parallel}	

\begin{lstlisting}
public class ParallelPriorAlgorithm {
	
	public List<List<Solution>> schedule(List<Event> events1, List<DateTimeClass> dtc1,
	List<Solution> externalExams, int maxTimePerDay,
	int solutionsMaxSize, int parallel) {
		List<int[]> solutions = new ArrayList<>();
		List<Event> events;
		List<Solution> external;
		List<DateTimeClass> dtc;
		Map<Event, DateTimeClass> externalMap = new HashMap<>();
		List<DateTimeClass> externalDtc = new ArrayList<>();
		int[] solution;
		int eventSize;
		int dtcSize;
		boolean[] ignoreWishes;
		events = new ArrayList<>();
		dtc = new ArrayList<>();
		
		events.addAll(events1);
		Collections.sort(events);
		Set<DateTimeClass> dtcSet = new HashSet<>(dtc1);
		//сначала те события, для которых определено время
		if (externalExams != null && externalExams.size() > 0) {
			external = externalExams;
			for (Solution sol : external) {
				Event e = sol.event;
				// System.out.println("ext: " + e);
				externalMap.put(e, sol.dtc);
				for (int i = 0; i < e.type.duration; i++) {
					DateTimeClass dtc2 = new DateTimeClass(sol.dtc.date, sol.dtc.time + i, sol.dtc.classroom);
					externalDtc.add(dtc2);
				}
				int i = events.size() - 1;
				while (i >= 0) {
					Event ev = events.get(i);
					if (e.course.equals(ev.course) && e.group.name.equals(ev.group.name)) {
						events.remove(ev);
						break;
					}
					i--;
				}
			}
			dtcSet.addAll(externalDtc);
		}
		Collections.sort(events);
		events.addAll(0, externalMap.keySet());
		dtc.addAll(dtcSet);
		Collections.sort(dtc);
		eventSize = events.size();
		dtcSize = dtc.size();
		solution = new int[dtcSize];
		for (int i = 0; i < dtcSize; i++) {
			solution[i] = -1;
		}
		for (Event event : externalMap.keySet()) {
			for (int i = 0; i < dtcSize; i++) {
				if (dtc.get(i).equals(externalMap.get(event))) {
					submitDateTimeClass(events, solution, events.indexOf(event), i);
					break;
				}
			}
		}
		ignoreWishes = new boolean[eventSize];
		ScheduleThread[] scheduleThreads = new ScheduleThread[parallel];
		Thread[] threads = new Thread[parallel];
		for (int i = 0; i < parallel; i++) {
			scheduleThreads[i] = new ScheduleThread(new ArrayList<>(events),
			new ArrayList<>(dtc), new HashMap<>(externalMap),
			new ArrayList<>(solutions), maxTimePerDay, solutionsMaxSize,
			solution.clone(), eventSize, dtcSize, ignoreWishes.clone(), (dtcSize * i / parallel));
			threads[i] = new Thread(scheduleThreads[i], "Thread " + (dtcSize * i / parallel));
			threads[i].start();
		}
		while (true) {
			int alive = parallel;
			for (int i = parallel - 1; i >= 0; i--) {
				if (!threads[i].isAlive()) {
					alive--;
					if (!scheduleThreads[i].solutions.isEmpty()) {
						solutions = scheduleThreads[i].solutions;
					}
				}
			}
			if (alive == 0 || !solutions.isEmpty()) {
				for (int j = 0; j < parallel; j++) {
					threads[j].interrupt();
				}
				break;
			}
		}
		return getSolutions(new ArrayList<>(solutions), events, dtc);
	}
	
	// бронинирование за событием помещения и времени
	private int[] submitDateTimeClass(List<Event> events, int[] solution, int event, int nextTime) {
		int duration = events.get(event).type.duration;
		for (int i = nextTime; i < nextTime + duration; i++) {
			solution[i] = event;
		}
		return solution;
	}
	
	private List<List<Solution>> getSolutions(List<int[]> solutions, List<Event> events, List<DateTimeClass> dtc) {
		if (solutions.size() == 0) return new ArrayList<>();
		List<List<Solution>> res = new ArrayList<>();
		for (int[] ints : solutions) {
			List<Solution> sh = new ArrayList<>();
			for (int j = 0; j < ints.length; j++) {
				if (ints[j] != -1) {
					sh.add(new Solution(events.get(ints[j]), dtc.get(j)));
				}
			}
			res.add(sh);
		}
		return res;
	}
}

class ScheduleThread implements Runnable {
	public List<int[]> solutions;
	List<Event> events;
	List<DateTimeClass> dtc;
	Map<Event, DateTimeClass> externalMap;
	int[] solution;
	int eventSize;
	int dtcSize;
	int start;
	int maxTimePerDay;
	int solutionsMaxSize;
	boolean[] ignoreWishes;
	
	public ScheduleThread(List<Event> events,
	List<DateTimeClass> dtc,
	Map<Event, DateTimeClass> externalMap,
	List<int[]> solutions,
	int maxTimePerDay,
	int solutionsMaxSize,
	int[] solution,
	int eventSize,
	int dtcSize,
	boolean[] ignoreWishes,
	int start) {
		this.solutions = solutions;
		this.events = events;
		this.dtc = dtc;
		this.start = start;
		this.externalMap = externalMap;
		this.solution = solution;
		this.eventSize = eventSize;
		this.dtcSize = dtcSize;
		this.maxTimePerDay = maxTimePerDay;
		this.solutionsMaxSize = solutionsMaxSize;
		this.ignoreWishes = ignoreWishes;
	}
	
	@Override
	public void run() {
		int externalSize = externalMap.keySet().size();
		int event = externalSize;
		if (externalSize == eventSize) {
			solutions.add(solution);
			return;
		}
		boolean f = false;
		while (event < eventSize) {
			int time = start;
			if (f) {
				time = getTime(event, dtcSize, solution);
			}
			f = true;
			solution = cleanEvent(event, dtcSize, solution);
			int nextTime = findNextStartTime(event, time, events, dtc, solution, dtcSize, maxTimePerDay, ignoreWishes);
			//переходим в режим игнорирования пожеланий преподавателя, если не смогли найти идеального решения
			if (!ignoreWishes[event] && nextTime == -1) {
				ignoreWishes[event] = true;
				nextTime = findNextStartTime(event, 0, events, dtc, solution, dtcSize, maxTimePerDay, ignoreWishes);
			}
			if (nextTime == -1) { //если не удалось найти другого подходящего DateTimeClass для этого события
				if (event == externalSize) {
					// System.out.println("END FIND SOL :(");
					return; //если речь о первом событии, то уже были перебраны все остальные варинты даже без учёта пожеланий преподавателя
				} else {
					ignoreWishes[event] = false;
					event--; // иначе возвращаемся к предыдущему событию и пробуем изменить для него DateTimeClass
				}
			} else {
				solution = submitDateTimeClass(events, solution, event, nextTime); //бронируем за событием время и аудиторию
				event++; // переходим к следующему событию
			}
			// если решение найдено, добавляем его в список решений и продолжаем искать решения
			if (event == eventSize) {
				solutions.add(solution.clone());
				if (solutions.size() >= solutionsMaxSize) return;
				event--;
			}
		}
		return;
		
	}
	
	private int[] cleanEvent(int event, int dtcSize, int[] solution) {
		for (int i = 0; i < dtcSize; i++) {
			if (solution[i] == event) solution[i] = -1;
		}
		return solution;
	}
	
	// бронинирование за событием помещения и времени
	private int[] submitDateTimeClass(List<Event> events, int[] solution, int event, int nextTime) {
		int duration = events.get(event).type.duration;
		for (int i = nextTime; i < nextTime + duration; i++) {
			solution[i] = event;
		}
		return solution;
	}
	
	// возвращает следующий свободный индекс DateTimeClass для указанного события
	private int getTime(int event, int dtcSize, int[] solution) {
		int min = -1;
		boolean contains = false;
		for (int i = 0; i < dtcSize - 1; i++) {
			// если аудитория в это время ещё никем не забронирована
			if (solution[i] == -1) {
				if (contains) {
					return i;
				} else if (min == -1) {
					min = i;
				}
			}
			// если это событие уже бронировало какое-то время и место
			if (solution[i] == event) {
				if (contains) return i;
				contains = true;
			}
		}
		// если последняя ячейка была занята текущим событием, то следующей для него нет
		if (contains) return -1;
		return min;
	}
	
	private int findNextStartTime(int event, int time, List<Event> events, List<DateTimeClass> dtc,
	int[] solution, int dtcSize, int maxTimePerDay, boolean[] ignoreWishes) {
		if (time < 0) return -1;
		Event ev = events.get(event);
		StudyGroup group = ev.group;
		Set<Teacher> teachers = ev.teacher;
		int duration = ev.type.duration - 1;
		LocalDate dateGroupEvent = null;
		Integer timeGroupEvent = null;
		if (ev.eventGroup != -1 && event != 0) {
			for (int i = 0; i < solution.length; i++) {
				if (solution[i] != -1 && solution[i] != event
				&& events.get(solution[i]).eventGroup.equals(ev.eventGroup)) {
					if (!(ev.group.name.equals(events.get(solution[i]).group.name) && Math.min(ev.type.countPerDay, events.get(solution[i]).type.countPerDay) < 2)) {
						dateGroupEvent = dtc.get(i).date;
					}
					if (ev.teacher.size() > 1) {
						timeGroupEvent = dtc.get(i).time;
					}
					break;
				}
			}
		}
		for (int i = time; i < dtcSize; i++) {
			DateTimeClass location = dtc.get(i);
			// если для события установлено, что оно должно проводиться в один день с другим
			if ((dateGroupEvent != null && !dateGroupEvent.equals(location.date))
			) {
				continue;
			}
			boolean teacherWishes = teachers.stream().mapToInt(teacher -> {
				if (
				teacher.date.stream().anyMatch(dt -> dt.equals(location.date))
				&& teacher.time.stream().anyMatch(t -> t.equals(location.time))
				&& teacher.time.stream().anyMatch(t -> t.equals(location.time + duration))) return 0;
				else return 1;
			}).sum() == 0;
			// в режиме игнорирования учитываем только события НЕ удовлетворяющие пожелания преподавателя
			if (ignoreWishes[event]) {
				teacherWishes = !teacherWishes;
			}
			
			boolean classRoom;
			if (ev.link != null && !ev.link.equals("")) {
				classRoom = location.classroom.num.equals(ev.link);
			} else if (ev.wishedClassroomNums != null && !ev.wishedClassroomNums.equals("")) {
				classRoom = ev.wishedClassroomNums.contains(location.classroom.num);
			} else {
				classRoom = ev.wishedClassroomType <= location.classroom.type;
			}
			if (classRoom //есть ли в аудитории нужное оборудование
			&& teacherWishes // подходит ли дата и время преподавателю
			&& solution[i] == -1 // аудитория в это время свободна
			//&& group.size <= location.classroom.size //влезает ли группа в аудиторию
			// не слишком ли сейчас поздно для начала проведения длительного события
			&& i + duration < dtcSize
			&& location.classroom.num.equals(dtc.get(i + duration).classroom.num)
			&& location.time + duration == dtc.get(i + duration).time // в массиве времени нет разрывов
			&& location.date.equals(dtc.get(i + duration).date)
			) {
				Map<String, Integer> teacherTime = new HashMap<>();
				int k = 0;
				int countPerDay = 0;
				while (k < dtcSize) {
					if (k == 0 || !dtc.get(k).date.equals(dtc.get(k - 1).date)) {
						countPerDay = 0;
					}
					// если время-место свободны, они не повлияют на ограничения
					if (solution[k] == -1) {
						k++;
						continue;
					}
					Event currentEvent = events.get(solution[k]);
					DateTimeClass currentLocation = dtc.get(k);
					// если проверяемая аудитория в это время занята
					if (location.date.equals(currentLocation.date) && currentLocation.time >= location.time && currentLocation.time < location.time + ev.type.duration && location.classroom.num.equals(currentLocation.classroom.num)) {
						break;
					}
					boolean success = false;
					// если у преподавателя в этот день уже были занятия
					if (currentLocation.date.equals(location.date)
					&& !(dateGroupEvent != null && ev.eventGroup.equals(currentEvent.eventGroup) && (ev.teacher.size() > 1))
					) {
						for (Teacher curT : currentEvent.teacher) {
							for (Teacher teacher : teachers) {
								if (curT.name.equals(teacher.name)) {
									if (!teacherTime.containsKey(teacher.name)) {
										teacherTime.put(teacher.name, 1);
									} else {
										teacherTime.compute(teacher.name, (key, v) -> v + 1);
									}
									// если преподаватель в это время ведёт другую аттестацию
									if (currentLocation.time >= location.time
									&& currentLocation.time <= location.time + duration) {
										success = true;
										break;
									}
									// если преподаватель уже достаточно отработал в этот день.
									if (teacherTime.get(teacher.name) > maxTimePerDay - duration) {
										success = true;
										break;
									}
								}
							}
							if (success) break;
						}
						if (success) {
							break;
						}
					}
					if (currentEvent.group.name.equals(group.name)) {
						//если у группы занятие прямо в этот момент
						if (location.date.equals(currentLocation.date) && location.time == currentLocation.time) {
							break;
						}
						//если в этот день уже были занятия у этой группы
						if (location.date.equals(currentLocation.date)) {
							countPerDay++;
							if (Math.min(ev.type.countPerDay * ev.type.duration, currentEvent.type.countPerDay * currentEvent.type.duration) > countPerDay) {
								break;
							}
						}
						//если до или после события есть другое событие, до которого меньше "отдыха", чем нужно
						if (location.date.minusDays(Math.max(ev.type.pauseBefore, currentEvent.type.pauseAfter)).compareTo(currentLocation.date) < 0
						&& location.date.plusDays(Math.max(ev.type.pauseAfter, currentEvent.type.pauseBefore)).compareTo(currentLocation.date) > 0) {
							break;
						}
					}
					k++;
				}
				if (k == dtcSize) {
					return i;
				}
			}
		}
		return -1;
	}
}
\end{lstlisting}