\chapter{Алгоритм поиска в ширину}\label{appendix-bfs}	

\begin{lstlisting}
public class DynamicScheduleAlgorithm {
	List<Event> events;
	List<DateTimeClass> dtc;
	List<List<Integer>>[][] solutionsMatrix;
	List<List<Integer>> solutions;
	int eventSize;
	int dtcSize;
	int maxTimePerDay;
	int maxSolutionSize;
	
	public DynamicScheduleAlgorithm(List<Event> events, List<DateTimeClass> dtc, int maxTimePerDay, int maxSolutionSize) {
		this.events = events;
		this.dtc = dtc;
		Collections.sort(this.dtc);
		this.eventSize = events.size();
		this.dtcSize = dtc.size();
		this.maxTimePerDay = maxTimePerDay;
		this.maxSolutionSize = maxSolutionSize;
		this.solutions = new ArrayList<>();
		this.solutionsMatrix = new ArrayList[eventSize][dtcSize];
		System.out.println("dtcSize " + dtcSize + " eventSize " + eventSize);
	}
	
	public List<List<Integer>> findSolutions() {
		//заполнение матрицы решений
		for (int i = 0; i < dtcSize; i++) {
			if (simpleCheck(0, i)) {
				solutionsMatrix[0][i] = new ArrayList<>();
				solutionsMatrix[0][i].add(Collections.singletonList(i));
			}
		}
		for (int i = 1; i < eventSize; i++) {
			int nullCount = 0;
			if (i > 1) {
				for (int j = 0; j < dtcSize; j++) {
					solutionsMatrix[i - 2] = null;
				}
			}
			System.gc();
			for (int j = 0; j < dtcSize; j++) {
				// если сама аудитория не подходит для проведения аттестации, нет смысла проверять её загруженность
				if (simpleCheck(i, j)) {
					for (int k = 0; k < dtcSize; k++) {
						if (solutionsMatrix[i - 1][k] != null) {
							for (List<Integer> path : solutionsMatrix[i - 1][k]) {
								if (pathsCheck(i, j, path)) {
									if (solutionsMatrix[i][j] == null) {
										solutionsMatrix[i][j] = new ArrayList<>();
									}
									List<Integer> newPath = new ArrayList<>(path);
									newPath.add(j);
									solutionsMatrix[i][j].add(newPath);
								}
							}
						}
					}
				}
				/* Если во время попытки разместить какую-то аттестацию выясняется, что её не возможно поставить,
				сразу возвращаем null, чтобы не обходить зря следующие аттестации
				*/
				if (solutionsMatrix[i][j] == null) {
					nullCount++;
				}
				if (nullCount == dtcSize) {
					System.out.println("NULL" + dtcSize);
					return null;
				}
			}
		}
		// собираем все возможные решения расстановки всех событий в один массив
		solutions = new ArrayList<>();
		for (int i = 0; i < dtcSize; i++) {
			if (solutionsMatrix[eventSize - 1][i] != null) {
				if (solutions.size() < maxSolutionSize)
				solutions.addAll(solutionsMatrix[eventSize - 1][i]);
			}
		}
		return solutions;
	}
	
	public List<Solution> getSolutions() {
		List<List<Solution>> res = new ArrayList<>();
		for (int i = 0; i < solutions.size(); i++) {
			List<Solution> sh = new ArrayList<>();
			for (int j = 0; j < events.size(); j++) {
				sh.add(new Solution(events.get(j), dtc.get(solutions.get(i).get(j))));
			}
			res.add(sh);
		}
		return res.get(0);
	}
	
	// проверка условий, которые завязаны на уже существующем расписании
	private boolean pathsCheck(int i, int j, List<Integer> path) {
		Event ev = events.get(i);
		DateTimeClass dateTimeClass = dtc.get(j);
		Set<Teacher> teachers = ev.teacher;
		StudyGroup group = ev.group;
		LocalDate pauseStart = dateTimeClass.date.minusDays(ev.type.pauseBefore);
		LocalDate pauseEnd = dateTimeClass.date.plusDays(ev.type.pauseAfter);
		int duration = ev.type.duration - 1;
		int groupsEventsToday = 1;
		for (int k = 0; k < path.size(); k++) {
			DateTimeClass d = dtc.get(path.get(k));
			Event e = events.get(k);
			if (d.date == dateTimeClass.date) {
				if (d.time >= dateTimeClass.time && d.time <= dateTimeClass.time + duration) {
					for (Teacher t : teachers) {
						for (Teacher t2 : e.teacher) {
							if (t.name.equals(t2.name)) {
								return false;
							}
						}
					}
				}
				if (group.name.equals(e.group.name)) {
					groupsEventsToday++;
				}
			}
			if (
			//если в этот день есть аттестации, которые по времени накладываются с текущей
			(d.date.equals(dateTimeClass.date) && (d.time + e.type.duration > dateTimeClass.time) && (dateTimeClass.time + duration > d.time))
			// если одна из предыдущих аттестаций попадает в окно отдыха до и после текущей аттестации
			|| (d.date.compareTo(pauseStart) > 0) && (d.date.compareTo(pauseEnd) < 0)
			// если у группы больше аттестаций этого типа, чем положено
			|| (groupsEventsToday > ev.type.countPerDay)
			) {
				return false;
			}
		}
		return true;
	}
	
	// проверка условий, которые не завязаны на уже существующем расписании
	private boolean simpleCheck(int i, int j) {
		Event ev = events.get(i);
		DateTimeClass location = dtc.get(j);
		StudyGroup group = ev.group;
		Set<Teacher> teachers = ev.teacher;
		int duration = ev.type.duration - 1;
		boolean teacherWishes = teachers.stream().mapToInt(teacher -> {
			if (
			teacher.date.stream().anyMatch(dt -> dt.equals(location.date))
			&& teacher.time.stream().anyMatch(t -> t.equals(location.time))
			&& teacher.time.stream().anyMatch(t -> t.equals(location.time + duration))) return 0;
			else return 1;
		}).sum() == 0;
		boolean classRoom;
		if (ev.link != null && !ev.link.equals("")) {
			classRoom = location.classroom.num.equals(ev.link);
		} else if (ev.wishedClassroomNums != null && !ev.wishedClassroomNums.equals("")) {
			classRoom = ev.wishedClassroomNums.contains(location.classroom.num);
		} else {
			classRoom = ev.wishedClassroomType <= location.classroom.type;
		}
		boolean res = classRoom //есть ли в аудитории нужное оборудование
		&& teacherWishes // подходит ли дата и время преподавателю
		&& group.size <= location.classroom.size //влезает ли группа в аудиторию
		// не слишком ли сейчас поздно для начала проведения длительного события
		&& i + duration < dtcSize
		&& location.classroom.num.equals(dtc.get(i + duration).classroom.num)
		&& location.time + duration == dtc.get(i + duration).time // в массиве времени нет разрывов
		&& dtc.get(i).date.equals(dtc.get(i + duration).date);
		return res;
	}
}
\end{lstlisting}