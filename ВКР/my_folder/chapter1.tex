\begin{comment}
\chapter{Анализ предметной области и обзор систем составления расписания} \label{ch1}

% не рекомендуется использовать отдельную section <<введение>> после лета 2020 года
%\section{Введение. Сложносоставное название первого параграфа первой главы для~демонстрации переноса слов в содержании} \label{ch1:intro}

Проблема составления расписания не нова, и существует множество средств, призванных упростить её решение. В интернете можно найти онлайн-календари с возможностью совместного редактирования, системы управления бизнес-процессами и программы генерации расписания для разного рода предприятий. Но далеко не все из них могут быть использованы в качестве полноценной системы составления расписания сессии для университета, поэтому в параграфе \ref{ch1:sec1} рассматриваются требования к системе составления расписания сессии СПбПУ.

 На рынке существует отдельная ниша систем составления расписания для ВУЗов, и в параграфе \ref{ch1:sec2} рассматриваются её русскоязычные представители, а в параграфе \ref{ch1:sec3} - зарубежные. Далее приведено сравнение таких систем и анализ их преимуществ и недостатков для решения конкретной задачи - составления предварительного расписания сессии в университете.

\section{Требования к системе составления расписания сессии} \label{ch1:sec1}

\subsection{Учёт семестрового расписания СПбПу}

Важным аспектом системы составления расписания сессии СПбПу является учёт занятости аудиторий и преподавателей. Бывают ситуации, когда сессия для некоторых учёбных групп начинается в тот момент, когда у других групп всё ещё проводятся семестровые занятия. Чтобы иметь возможность составлять расписание сессии, которое не ставит проведение аттестаций в занятые по семестровому расписанию аудитории, необходимо обращаться c помощью API к официальному расписанию занятий  \cite{ruz}. 

\subsection{Роли пользователей системы}
Одним из требований к системе сбора сведений является выделение двух ролей пользователей:
\begin{itemize}
	\item Администратор - пользователь, который сообщает системе сведения о сессии, аудиториях и о плане экзаменов, а также инициирует процесс генерации расписания.
	\item Преподаватель - пользователь, который сообщает системе о собственных пожеланиях к проведению своих экзаменов.
\end{itemize}

Рассмотрим возможности этих типов пользователей подробнее. На диаграмме на рисунке \ref{fig:usecse1} показаны возможности преподавателя.

Преподаватель может поучаствовать в составлении расписания, указав даты, дни недели и время, когда он доступен или не доступен для проведения аттестаций, необходимые типы аудиторий, своих ассистентов и пожелания по компоновки групп по дням.

\begin{minipage}{\textwidth}
	\centering
	\vspace{\mfloatsep} % интервал  	
	\includegraphics[scale=0.6, keepaspectratio=true] {my_folder/images/usecase1}
	\captionof{figure}{Use case диаграмма ППС}\label{fig:usecse1}  
	\vspace{\mfloatsep} % интервал  	
\end{minipage}

Возможности администратора показаны на диаграмме на рисунке \ref{fig:usecse2}.

Администратор заполняет общие сведения о сессии, такие как даты, время, доступные аудитории с указанием, сколько в них мест и есть ли там проектор и компьютеры и предстоящие экзамены по группам и преподавателям. Также, могут быть указаны аттестации, для которых уже определено время и место проведение. Это актуально, например, для событий, проводимых другими подразделениями университета. В возможности администратора помимо вышесказанного входит генерация формы сбора предпочтений преподавателей. 

\begin{minipage}{\textwidth}
	\centering
	\vspace{\mfloatsep} % интервал  	
	\includegraphics[keepaspectratio=true,scale=0.6] {my_folder/images//usecase2}
	\captionof{figure}{Use case диаграмма администратора}\label{fig:usecse2}  
	\vspace{\mfloatsep} % интервал  	
\end{minipage}

Таким образом, когда администратор заполнит все вышеперечисленные поля формы, в системе уже будет минимальный набор данных, необходимый для составления расписания. 
Далее, преподаватели могут заполнять свои формы с пожеланиями к своему расписанию. Незаполненная преподавателем форма не будет являться проблемой для системы. Пустая форма по умолчанию приравнивается к готовности преподавателя проводить экзамены и зачёты в любой день, в любое время. 

\subsection{Модель входных данных} \label{ch1:sec1:sub3}
Исходя из данных, которые необходимо учитывать при составлении расписания сессии была спроектирована модель данных, соответсвующая схеме базы данных для их хранения. Она представлена на схеме \ref{fig:bd}.

\begin{minipage}{\textwidth}
	\centering
	\vspace{\mfloatsep} % интервал  	
	\includegraphics[ keepaspectratio=true, scale=0.4] {my_folder/images/bd}
	\captionof{figure}{Схема базы данных}\label{fig:bd}  
	\vspace{\mfloatsep} % интервал  	
\end{minipage}

Сущности схемы:

\begin{itemize}
	\item Teacher - таблица преподавателей.
	\begin{itemize}
		\item id - уникальный идентификатор
		\item name - имя
		\item prior - приоритет
	\end{itemize} 
	
	\item TeacherExamDate - таблица дат, которые преподаватель указывает, как доступные для проведения аттестаций. 
	\begin{itemize}
		\item id - уникальный идентификатор
		\item date - дата
		\item teacherId - id преподавателя
	\end{itemize} 
	
	\item TeacherExamTime - таблица времени, которое преподаватель указывает, как доступное для проведения аттестаций.
	\begin{itemize}
		\item id - уникальный идентификатор
		\item time - время
		\item teacherId - id преподавателя
	\end{itemize} 
		
	\item StudyGroup - таблица учебных групп.
	\begin{itemize}
		\item id - уникальный идентификатор
		\item level - курс
		\item name - номер
		\item size - кол-во студентов
	\end{itemize} 

	\item AttestationType - таблица типов аттестаций.
	\begin{itemize}
		\item id - уникальный идентификатор
		\item countPerDay - количество в день
		\item duration - длительность в часах
		\item pauseAfter -пауза в днях после аттестации
		\item pauseBefore - пауза в днях до аттестации
		\item type - название типа
	\end{itemize} 

	\item Classroom - таблица аудиторий с указанием есть ли там компьютеры или проектор и количеством мест.
	\begin{itemize}
		\item id - уникальный идентификатор
		\item classroom - номер кабинета
		\item computers - наличие компьютеров
		\item projector - наличие проектора
		\item sitscount - кол-во мест в аудитории
	\end{itemize} 

	\item SessionDate - таблица дат, в которые проводится сессия. Такой способ хранения данных о датах сессии был выбран вместо хранения даты начала и окончания, потому что зимняя сессия может состоять из нескольких интервалов, прерванных например новогодними праздниками.
	\begin{itemize}
		\item id - уникальный идентификатор
		\item date - дата
	\end{itemize} 

	\item SessionTime - таблица часов для проведения аттестаций.
	\begin{itemize}
		\item id - уникальный идентификатор
		\item time - время
	\end{itemize} 	

	\item ExternalExamLink - ссылки на google-таблицы, с аттестациями, для которых назначено время и место проведения. Они могут использоваться, например, для учёта аттестаций, которые назначаются другими подразделениями университета
\begin{itemize}
	\item id - уникальный идентификатор
	\item link - ссылка
\end{itemize} 

	\item Curriculum - таблица для хранения учебного плана.
\begin{itemize}
	\item id - уникальный идентификатор
	\item cource - название дисциплины
	\item semester - семестр, в который проводится аттестации
	\item specCode - специальность
	\item type - тип аттестации
\end{itemize} 

	\item Exam - таблица аттестаций, данные о которых вводятся вместе с данными о преподавателях. Используется при формировании расписании сессии заочников, когда по данным из расписания ruz, нельзя определить, кто должен проводить аттестации.
\begin{itemize}
	\item id - уникальный идентификатор
	\item cource - название дисциплины
	\item groupName - группа
	\item levels - семестр
	\item specName - специальность
	\item teacherName - преподаватель
	\item type - тип аттестации
\end{itemize} 
\end{itemize}

Описанные выше сущности соотносятся с данными, которые необходимо получать от пользователей. В процессе работы алгоритма составления расписания в персистентное хранилище записываются также следующие сущности: 
\begin{itemize}
	\item ResultLink - ссылки на google-таблицы, со сгенерированными ранее вариантами расписаний 
	\begin{itemize}
		\item id - уникальный идентификатор
		\item link - ссылка
	\end{itemize} 

	\item DateTimeClass - таблица доступных окон - дата+время+аудитория. В ней хранятся окна для проведения аттестаций, из которых исключены варианты, занятые другими занятиями.
	\begin{itemize}
		\item id - уникальный идентификатор
		\item date - дата
		\item time - время
		\item classroomId - id аудитории
	\end{itemize} 

	\item Event - таблица аттестаций, которые необходимо провести.
\begin{itemize}
	\item id - уникальный идентификатор
	\item cource - название дисциплины
	\item wishedClassroomType - необходимый тип аудитории
	\item teacherId - id преподавателя
	\item groupId - id учебной группы
\end{itemize} 
\end{itemize}


\section{Обзор российских систем составления расписания} \label{ch1:sec2}

\subsection{1С: ХроноГраф Расписание} 
Фирма «1С», занимающаяся разработкой ПО для бизнеса и образования в качестве системы для автоматизации учебного планирования и составления расписания в разного рода организациях предлагает свою программу «1С: ХроноГраф Расписание» \cite{1с}.
«1С: ХроноГраф Расписание» позволяет:
\begin{itemize}
	\item cоставлять понедельное расписание организации или отдельных её подразделений;
	\item задавать периоды обучения с учётом нерабочих дней, каникул и разбиением на четные и нечётные недели;
	\item создавать черновое расписание, используя функцию «Предварительный расчёт».
\end{itemize}

Основной проблемой данной программы является несовместимость с другими платформами. «1С: ХроноГраф Расписание» - однопользовательская программа, и нельзя интегрировать её с web-приложением для возможности сбора данных напрямую от пользователей. Сложность составления расписания сессии в этой системе обуславливается также её ориентированностью на составление расписания по неделям без учёта специфики проведения аттестаций.

\subsection{Avtor (АВТОРасписание)} 
Программа «АВТОРасписание»  \cite{avtor} имеет несколько версий для различных учебных заведений: общеобразовательных школ, колледжей, техникумов, профессиональных училищ и ВУЗов. Это позволяет в подстроиться под специфику расписания конкретного типа образовательного учреждения, что является одним из её конкурентых преимуществ.

«АВТОРасписание» имеет достаточно широкий спектр применений. Этот программный продукт позволяет
\begin{itemize}
	\item cоставлять понедельное расписание для учебных групп с минимальным количеством окон;
	\item cоставлять расписание преподавателей с минимальным количеством окон;
	\item оптимально размещать занятия по аудиториям, учитывая их вместимость и оснащённость необходимым оборудованием;
	\item учитывать пожелания сотрудников к своему расписанию;
	\item разделять учебные группы на подгруппы;
	\item вносить ручные корректировки в расписание.
\end{itemize}

Преимуществом этой программы помимо прочего является возможность публиковать расписание обучающихся и преподавателей из самой системы «Автор» на сайте, внутреннем портале или на мультимедийных стендах образовательной организации. Но при этом импорт данных всё ещё производится вручную диспетчером, что не очень удобно для учебного заведения с большим штабом сотрудников, которые сами могли бы вносить свои пожелания в систему.

\subsection{Галактика Расписание учебных занятий}
«Галактика Расписание учебных занятий» - часть системы управления ВУЗом той же корпорации Галактика  \cite{galaktica}. Этот программный продукт позволяет составлять расписание в ВУЗе, а также:

\begin{itemize}
	\item вычислять несколько десятков показателей эффективности расписаний;
	\item оптимально размещать занятия по аудиториям, учитывая их вместимость и оснащённость необходимым оборудованием;
	\item учитывать приоритет преподавателей, учебных групп и дисциплин;
	\item контролировать пересечение расписаний для преподавателей, учебных групп и подгрупп во избежание «накладок»;
	\item контролировать длительность занятий;
	\item вручную бронировать аудиторный фонд;
	\item учитывать план изучения дисциплин для выстраивания их в правильном порядке.
\end{itemize}

«Галактика Расписание учебных занятий» - серьёзный инструмент для формирования расписания в высших учебных заведениях, учитывающий множество факторов при его составлении и имеющий удобную систему отчётности. На данный момент эта программа наиболее полно решает проблему автоматической генерации расписания российских ВУЗов, но и она не имеет интерфейса для прямого импорта пожеланий преподавателей прямо в систему. Компания «Галактика» помимо прочего предлагает техническое сопровождение своего ПО, но это учитывается при расчёте стоимости лицензии на использование программы.
	
\section{Обзор зарубежных систем составления расписания} \label{ch1:sec3}	

\subsection {Apereo UniTime}
UniTime от компании Apereo \cite{unitime} - система автоматического создания расписания западных высших учебных заведений. Она учитывает, что студенты могут выбирвать себе индивидуальный набор курсов, чего не происходит в российских ВУЗах, где обучение происходит по плану образовательных программ направлений.

UniTime даёт возможность:
\begin{itemize}
	\item автоматически генерировать расписание курсов и экзаменов;
	\item минимизировать конфликты студенческих курсов;
	\item вносить ручные корректировки в расписание.
\end{itemize}

Эта программа имеет понятный web-интерфейс и может быть интегрирована в другую систему, но она не позволяет преподавателям вносить данные о своей занятости, чтобы учесть их при составлении расписания. Неприспособленность программы под составление расписания для групп, а не для конкретных студентов делает её менее удобной, чем российские аналоги.

\subsection {Lantiv Scheduling Studio} 
Программа «Scheduling Studio» \cite{lantiv} от компании Lantiv представляет собой систему совместной работы над расписанием и реализует следующие задачи:

\begin{itemize}
	\item совместный доступ к редактированию расписания ВУЗа;
	\item оффлайн редактирование с возможностью синхронизации после появления в сети;
	\item цветовое выделение накладок расписания;
	\item cоставление расписания на различные временные периоды: неделя, семестр, четверть, год;
	\item копирование составленных элементов расписания на другие периоды.
\end{itemize}

Данный программный продукт имеет приятный и понятный интерфейс, но не имеет модуля автоматической генерации расписания, из-за чего основная часть работы всё ещё ложится на плечи диспетчеров. «Scheduling Studio» удобно использовать для составления нетривиального расписания, которое меняется от недели к неделе и плохо вписывается в шаблон школьного расписания или расписания учебных занятий ВУЗа, например. Но для составления расписания сессии требуется большая степень автоматизации, чем предлагается этим ПО.

\section{Выводы} \label{ch1:conclusion}
Сведения о возможностях каждого из описанного в параграфах	\ref{ch1:sec2} и \ref{ch1:sec3} сведём в таблицу \ref{tab:1.4.1}.
\begin{table} [htbp]
	\centering\small
	\caption{Сравнение систем составления расписания}%
	\label{tab:1.4.1}	
	\begin{tabular}{|p{0.18\linewidth}|p{0.1\linewidth}|p{0.15\linewidth}|p{0.1\linewidth}|p{0.08\linewidth}|p{0.1\linewidth}|p{0.1\linewidth}|}
		\hline
		&Учитывает пожелания ППС&Интегрируется с сайтами ВУЗов&Имеет возможность задавать нетривиальное расписание&Плата за использование&Генерация предварительного расписания&Открытый исходный код\\
		\hline
		1С: ХроноГраф Расписание&+&-&-&+&+&-\\ \hline
		Avtor&+&-&+&+&+&-\\ \hline
		Галактика Расписание учебных занятий&+&+&+&+&+&-\\ \hline
		Apereo UniTime&-&+&+&-&+&+\\ \hline
		Lantiv Scheduling Studio&-&-&+&+&-&-\\ \hline	
	\end{tabular}
\end{table}

Видно, что среди систем составления расписания для составления предварительного расписания сессии лучше всего могла бы подойти программа «Галактика Расписание учебных занятий», так как она удовлетворяет большинству требований, но при этом она, как и почти всё представленное в таблице ПО, требует оплату за использование. Также среди представленных программных продуктов все, кроме одного, не предоставляют открытый доступ к исходному коду. 
Таким образом, в рамках данной работы необходимо было разработать программу, удовлетворяющую всем перечисленным в таблице критериям.
\end{comment}
\chapter{Анализ систем составления расписания в ВУЗах} \label{ch1}
%%ПА: проще без ``Анализ предметной области и обзор''

Проблема составления расписания не нова, и существует множество средств, призванных упростить её решение. В интернете можно найти онлайн-календари с возможностью совместного редактирования, системы управления бизнес-процессами и программы генерации расписания для разного рода предприятий. Но далеко не все из них могут быть использованы в качестве полноценной системы составления расписания сессии для университета, поэтому в параграфе \ref{ch1:sec1} рассматриваются требования к системе составления расписания сессии СПбПУ.

На рынке существует отдельная ниша систем составления расписания для ВУЗов, и в параграфе \ref{ch1:sec2} рассматриваются её русскоязычные представители, а в параграфе \ref{ch1:sec3} - зарубежные. Далее приведено сравнение таких систем и анализ их преимуществ и недостатков для решения конкретной задачи - составления предварительного расписания сессии в университете.

\section{Требования к системе составления расписания сессии} \label{ch1:sec1}

\subsection{Учёт семестрового расписания СПбПУ}

Важным аспектом системы составления расписания сессии СПбПУ является учёт занятости аудиторий и преподавателей. Бывают ситуации, когда сессия для некоторых учебных групп начинается в тот момент, когда у других групп всё ещё проводятся семестровые занятия. Чтобы иметь возможность составлять расписание сессии, которое не ставит проведение аттестаций в занятые по семестровому расписанию аудитории, необходимо обращаться c помощью API к официальному расписанию занятий  \cite{ruz}. 

\subsection{Роли пользователей системы}
Одним из требований к системе сбора сведений является выделение двух ролей пользователей:
\begin{itemize}
	\item Администратор - пользователь, который сообщает системе сведения о сессии, аудиториях и о плане экзаменов, а также инициирует процесс генерации расписания.
	\item Преподаватель - пользователь, который сообщает системе о собственных пожеланиях к проведению своих экзаменов.
\end{itemize}

Рассмотрим возможности этих типов пользователей подробнее. На диаграмме на рисунке \ref{fig:usecse1} показаны возможности преподавателя.
Преподаватель может участвовать в составлении расписания, указав даты, дни недели и время, когда он доступен или не доступен для проведения аттестаций, необходимые типы аудиторий, своих ассистентов и пожелания по компоновки групп по дням.

\begin{minipage}{\textwidth}
	\centering
	\vspace{\mfloatsep} % интервал  	
	\includegraphics[scale=0.6, keepaspectratio=true] {my_folder/images/usecase1}
	\captionof{figure}{Use case диаграмма ППС}\label{fig:usecse1}  
	\vspace{\mfloatsep} % интервал  	
\end{minipage}

Возможности администратора показаны на диаграмме на рисунке \ref{fig:usecse2}.
Администратор заполняет общие сведения о сессии, такие как даты, время, доступные аудитории с указанием, сколько в них мест и есть ли там проектор и компьютеры и предстоящие экзамены по группам и преподавателям. Также, могут быть указаны аттестации, для которых уже определено время и место проведение. Это актуально, например, для событий, проводимых другими подразделениями университета. В возможности администратора помимо вышесказанного входит генерация формы сбора предпочтений преподавателей. 

\begin{minipage}{\textwidth}
	\centering
	\vspace{\mfloatsep} % интервал  	
	\includegraphics[keepaspectratio=true,scale=0.6] {my_folder/images//usecase2}
	\captionof{figure}{Use case диаграмма администратора}\label{fig:usecse2}  
	\vspace{\mfloatsep} % интервал  	
\end{minipage}

Таким образом, когда администратор заполнит все вышеперечисленные поля формы, в системе уже будет минимальный набор данных, необходимый для составления расписания. 
Далее, преподаватели могут заполнять свои формы с пожеланиями к своему расписанию. Незаполненная преподавателем форма не будет являться проблемой для системы. Пустая форма по умолчанию приравнивается к готовности преподавателя проводить экзамены и зачёты в любой день, в любое время. 

\section{Обзор российских систем составления расписания} \label{ch1:sec2}

\subsection{1С: ХроноГраф Расписание} 
Фирма «1С», занимающаяся разработкой ПО для бизнеса и образования, в качестве системы для автоматизации учебного планирования и составления расписания в разного рода организациях предлагает свою программу «1С: ХроноГраф Расписание» \cite{1с}.
«1С: ХроноГраф Расписание» позволяет:
\begin{itemize}
	\item cоставлять понедельное расписание организации или отдельных её подразделений;
	\item задавать периоды обучения с учётом нерабочих дней, каникул и разбиением на четные и нечётные недели;
	\item создавать черновое расписание, используя функцию «Предварительный расчёт».
\end{itemize}

Основной проблемой данной программы является несовместимость с другими платформами. «1С: ХроноГраф Расписание» - однопользовательская программа, и нельзя интегрировать её с web-приложением для возможности сбора данных напрямую от пользователей. Сложность составления расписания сессии в этой системе обуславливается также её ориентированностью на составление расписания по неделям без учёта специфики проведения аттестаций.

\subsection{Avtor}% сокращения не вводят в заголовках! 
Программа Avtor («АВТОРасписание»)  \cite{avtor} имеет несколько версий для различных учебных заведений: общеобразовательных школ, колледжей, техникумов, профессиональных училищ и ВУЗов. Это позволяет в подстроиться под специфику расписания конкретного типа образовательного учреждения, что является одним из её конкурентых преимуществ.

«АВТОРасписание» имеет достаточно широкий спектр применений. Этот программный продукт позволяет
\begin{itemize}
	\item cоставлять понедельное расписание для учебных групп с минимальным количеством окон;
	\item cоставлять расписание преподавателей с минимальным количеством окон;
	\item оптимально размещать занятия по аудиториям, учитывая их вместимость и оснащённость необходимым оборудованием;
	\item учитывать пожелания сотрудников к своему расписанию;
	\item разделять учебные группы на подгруппы;
	\item вносить ручные корректировки в расписание.
\end{itemize}

Преимуществом этой программы помимо прочего является возможность публиковать расписание обучающихся и преподавателей из самой системы «Автор» на сайте, внутреннем портале или на мультимедийных стендах образовательной организации. Но при этом импорт данных всё ещё производится вручную диспетчером, что не очень удобно для учебного заведения с большим штабом сотрудников, которые сами могли бы вносить свои пожелания в систему.

\subsection{Галактика Расписание учебных занятий}
%ПА: непонятно почему ``той же'' и почему это ``корпорация''?)
«Галактика Расписание учебных занятий» - часть системы управления ВУЗом организации Галактика  \cite{galaktica}. Этот программный продукт позволяет составлять расписание в ВУЗе, а также:

\begin{itemize}
	\item вычислять несколько десятков показателей эффективности расписаний;
	\item оптимально размещать занятия по аудиториям, учитывая их вместимость и оснащённость необходимым оборудованием;
	\item учитывать приоритет преподавателей, учебных групп и дисциплин;
	\item контролировать пересечение расписаний для преподавателей, учебных групп и подгрупп во избежание «накладок»;
	\item контролировать длительность занятий;
	\item вручную бронировать аудиторный фонд;
	\item учитывать план изучения дисциплин для выстраивания их в правильном порядке.
\end{itemize}

«Галактика Расписание учебных занятий» - серьёзный инструмент для формирования расписания в высших учебных заведениях, учитывающий множество факторов при его составлении и имеющий удобную систему отчётности. На данный момент эта программа наиболее полно решает проблему автоматической генерации расписания российских ВУЗов, но и она не имеет интерфейса для прямого импорта пожеланий преподавателей прямо в систему. Компания «Галактика» помимо прочего предлагает техническое сопровождение своего ПО, но это учитывается при расчёте стоимости лицензии на использование программы.

Составление расписания в СПбПУ производится при активном использовании данной программы.

\section{Обзор зарубежных систем составления расписания} \label{ch1:sec3}	

\subsection {Apereo UniTime}
UniTime от компании Apereo \cite{unitime} - система автоматического создания расписания западных высших учебных заведений. Она учитывает, что студенты могут выбирать себе индивидуальный набор курсов, и составляет индивидуальное расписание именно для студаентов, а не для учебных групп. 

UniTime даёт возможность:
\begin{itemize}
	\item автоматически генерировать расписание курсов и экзаменов;
	\item минимизировать конфликты студенческих курсов;
	\item вносить ручные корректировки в расписание.
\end{itemize}

Эта программа имеет понятный web-интерфейс и может быть интегрирована в другую систему, но она не позволяет преподавателям вносить данные о своей занятости, чтобы учесть их при составлении расписания. Неприспособленность программы под составление расписания для групп, а не для конкретных студентов делает её менее удобной, чем российские аналоги.

\subsection {Lantiv Scheduling Studio} 
Программа «Scheduling Studio» \cite{lantiv} от компании Lantiv представляет собой систему совместной работы над расписанием и реализует следующие задачи:

\begin{itemize}
	\item совместный доступ к редактированию расписания ВУЗа;
	\item оффлайн редактирование с возможностью синхронизации после появления в сети;
	\item цветовое выделение накладок расписания;
	\item cоставление расписания на различные временные периоды: неделя, семестр, четверть, год;
	\item копирование составленных элементов расписания на другие периоды.
\end{itemize}

Данный программный продукт имеет приятный и понятный интерфейс, но не имеет модуля автоматической генерации расписания, из-за чего основная часть работы всё ещё ложится на плечи диспетчеров. «Scheduling Studio» удобно использовать для составления нетривиального расписания, которое меняется от недели к неделе и плохо вписывается в шаблон школьного расписания или расписания учебных занятий ВУЗа, например. Но для составления расписания сессии требуется большая степень автоматизации, чем предлагается этим ПО.

\section{Выводы} \label{ch1:conclusion}
Сведения о возможностях каждого из описанного в параграфах	\ref{ch1:sec2} и \ref{ch1:sec3} сведём в таблицу \ref{tab:1.4.1}.
\begin{table} [htbp]
	\centering\small
	\caption{Сравнение систем составления расписания}%
	\label{tab:1.4.1}	
	\begin{tabular}{|p{0.18\linewidth}|p{0.1\linewidth}|p{0.15\linewidth}|p{0.1\linewidth}|p{0.08\linewidth}|p{0.1\linewidth}|p{0.1\linewidth}|}
		\hline
		&Учитывает пожелания ППС&Интегрируется с сайтами ВУЗов&Имеет возможность задавать нетривиальное расписание&Плата за использование&Генерация предварительного расписания&Открытый исходный код\\
		\hline
		1С: ХроноГраф Расписание&+&-&-&+&+&-\\ \hline
		Avtor&+&-&+&+&+&-\\ \hline
		Галактика Расписание учебных занятий&+&+&+&+&+&-\\ \hline
		Apereo UniTime&-&+&+&-&+&+\\ \hline
		Lantiv Scheduling Studio&-&-&+&+&-&-\\ \hline	
	\end{tabular}
\end{table}

Видно, что среди систем составления расписания для составления предварительного расписания сессии лучше всего могла бы подойти программа «Галактика Расписание учебных занятий», так как она удовлетворяет большинству требований, но при этом она, как и почти всё представленное в таблице ПО, требует оплату за использование. Также среди представленных программных продуктов все, кроме одного, не предоставляют открытый доступ к исходному коду. Рассматривался вариант с доработкой кода Apereo UniTime, но слишком многое в нём было завязано на индивидуальные графики учёбы студентов, а наиболее подходящим расписанием считалось то, которое минимизирует пересечение курсов одного учащегося.
В рамках данной работы необходимо было разработать программу, удовлетворяющую всем перечисленным в таблице критериям.