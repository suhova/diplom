\chapter*{Заключение} \label{ch-conclusion}
\addcontentsline{toc}{chapter}{Заключение}	% в оглавление 

В результате выполнения данной работы мною были сравнены существующие российские и зарубежные системы составления расписания, изучены алгоритмы составления и оптимизации расписания и подходы к этому процессу, спроектирована и реализована система составления предварительного расписания сессии СПбПУ. 

Созданный программный продукт автоматизирует сбор требований преподавателей к проведению аттестаций, получает данные о проведении занятий и занятости аудиторий с официального сайта расписания СПбПУ и генерирует предварительное расписание с учётом физических и нормативных ограничений, а также с учётом полученных пожеланий профессорско-преподавательского состава. Помимо прочего, он позволяет редактировать расписание с использованием google-таблиц и учитывать занятия с уже установленным временем и аудиторией, что даёт возможность экспериментировать с результатами.

Система была неоднократно протестирована и на этапе апробирования показала, что может быть использована не только для составления расписания аттестаций, но и для других не повторяющихся из недели в неделю событий, как например консультации ВКР, установочные занятия студентов заочной формы обучения или экзамены дополнительной сессии, что демонстрирует её широкий спектр применений.