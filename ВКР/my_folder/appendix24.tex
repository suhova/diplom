\chapter{Замеры времени и памяти} \label{appendix-timetest}

\begin{lstlisting}

public class TimeTest {
	private List<DateTimeClass> generateDTC(int datecount, int classcount) {
		LocalDate d1 = LocalDate.parse("2021-01-01");
		List<LocalDate> dateList = new ArrayList<>();
		for (int i = 0; i < datecount; i++) {
			dateList.add(d1.plusDays(i));
		}
		List<Integer> time = List.of(8, 9, 10, 11, 12, 13, 14, 15, 16, 17);
		List<Classroom> classrooms = new ArrayList<>();
		for (int i = 0; i < classcount; i++) {
			classrooms.add(new Classroom("c" + i, 10, 10));
		}
		return SessionService.createListDTC(dateList, time, classrooms);
	}
	
	private List<Event> generateEvents(int teacherCount, int teacherDateCount, int countGroups, int countEventsByGroup, int attDuration) {
		List<Teacher> teachers = new ArrayList<>();
		LocalDate d1 = LocalDate.parse("2021-01-01");
		for (int i = 0; i < teacherCount; i++) {
			Teacher teacher = new Teacher("t" + i);
			List<LocalDate> dateList = new ArrayList<>();
			for (int j = 0; j < teacherDateCount; j++) {
				dateList.add(d1.plusDays(j));
			}
			teacher.time = List.of(8, 9, 10, 11, 12, 13, 14, 15, 16, 17, 18);
			teacher.date = dateList;
			for (int j = 0; j < countEventsByGroup * countGroups / teacherCount; j++) {
				teachers.add(teacher);
			}
			Collections.shuffle(teachers);
		}
		List<StudyGroup> groups = new ArrayList<>();
		for (int i = 0; i < countGroups; i++) {
			groups.add(new StudyGroup("gr" + i, 1, 1));
		}
		AttestationType att = new AttestationType("Экз", 0, 0, attDuration, 3);
		List<Event> events = new ArrayList<>();
		for (int i = 0; i < countGroups; i++) {
			for (int j = 0; j < countEventsByGroup; j++) {
				events.add(new Event(groups.get(i), Set.of(teachers.get(countEventsByGroup * i + j)), att, "course" + i * countEventsByGroup + j));
			}
		}
		return events;
	}
	
	private long[] getTime(List<DateTimeClass> dtc, List<Event> events) {
		int n = 10;
		long time = 0;
		long mem = 0;
		long start = System.nanoTime();
		for (int i = 0; i < n; i++) {
			SchedulePriorAlgorithm algorithm = new SchedulePriorAlgorithm(events, dtc, new ArrayList<>(), 8, 1000);
			algorithm.findSolutions();
			long finish = System.nanoTime();
			time += finish - start;
			System.gc();
			mem += (Runtime.getRuntime().totalMemory() - Runtime.getRuntime().freeMemory());
		}
		return new long[]{(time / n) / 1000000, mem/n};
	}
	private long[] getParallelTime(List<DateTimeClass> dtc, List<Event> events) {
		int n = 10;
		long time = 0;
		long mem = 0;
		long start = System.nanoTime();
		for (int i = 0; i < n; i++) {
			ParallelPriorAlgorithm algorithm = new ParallelPriorAlgorithm();
			algorithm.schedule(events, dtc, new ArrayList<>(), 8, 1000, 2);
			long finish = System.nanoTime();
			time += finish - start;
			System.gc();
			mem += (Runtime.getRuntime().totalMemory() - Runtime.getRuntime().freeMemory());
		}
		return new long[]{(time / n) / 1000000, mem/n};
	}
	
	private long[] getTimeDynamic(List<DateTimeClass> dtc, List<Event> events) {
		int n = 10;
		long time = 0;
		long mem = 0;
		long start = System.nanoTime();
		for (int i = 0; i < n; i++) {
			DynamicScheduleAlgorithm algorithm = new DynamicScheduleAlgorithm(events, dtc, 8, 1);
			algorithm.findSolutions();
			algorithm.getSolutions();
			long finish = System.nanoTime();
			time += finish - start;
			System.gc();
			mem += (Runtime.getRuntime().totalMemory() - Runtime.getRuntime().freeMemory());
			
		}
		return new long[]{(time / n) / 1000000, mem/n};
	}
	
	private List<Event> resize(List<Event> events) {
		List<Event> res = new ArrayList<>();
		res.addAll(events);
		for (int i = 0; i < events.size(); i++) {
			Set<Teacher> set = new HashSet<>();
			events.get(i).teacher.forEach(x -> {
				Teacher t = new Teacher(x.name + "c");
				t.time = x.time;
				t.date = x.date;
				set.add(t);
			});
			res.add(new Event(new StudyGroup(events.get(i).group.name + "c", 10, 1),
			set, events.get(i).type, events.get(i).course + "c"));
		}
		return res;
	}
	
	@Test
	public void time1() {
		long[][] res = new long[20][4];
		List<Event> events = generateEvents(5, 50, 5, 3, 2);
		List<List<Event>> e = new ArrayList<>();
		List<Event> events1 = resize(events);
		List<Event> events2 = resize(events1);
		List<Event> events3 = resize(events2);
		e.add(events);
		e.add(events1);
		e.add(events2);
		e.add(events3);
		List<List<DateTimeClass>> d = new ArrayList<>();
		d.add(generateDTC(10, 5));
		d.add(generateDTC(20, 10));
		d.add(generateDTC(30, 15));
		d.add(generateDTC(50, 25));
		d.add(generateDTC(50, 50));
		for (int i = 0; i < d.size(); i++) {
			for (int j = 0; j < e.size(); j++) {
				long[] t = getTime(d.get(i), e.get(j));
				res[i][j] = t[0];
				res[i+5][j] = t[1];
			}
		}
		System.gc();
		for (int i = 0; i < d.size(); i++) {
			for (int j = 0; j < e.size(); j++) {
				long[] t = getParallelTime(d.get(i), e.get(j));
				res[i+10][j] = t[0];
				res[i+15][j] = t[1];
			}
		}
		System.out.println(googleRes(res));
	}
	
	@Test
	public void time2() {
		long[][] res = new long[20][4];
		List<Event> events = generateEvents(1, 15, 1, 1, 2);
		List<List<Event>> e = new ArrayList<>();
		List<Event> events1 = resize(events);
		List<Event> events2 = resize(events1);
		List<Event> events3 = resize(events2);
		e.add(events);
		e.add(events1);
		e.add(events2);
		e.add(events3);
		
		List<List<DateTimeClass>> d = new ArrayList<>();
		d.add(generateDTC(3, 3));
		d.add(generateDTC(4, 4));
		d.add(generateDTC(5, 5));
		d.add(generateDTC(6, 6));
		d.add(generateDTC(7, 7));
		for (int i = 0; i < d.size(); i++) {
			for (int j = 0; j < e.size(); j++) {
				long[] t = getTimeDynamic(d.get(i), e.get(j));
				res[i+10][j] = t[0];
				res[i+15][j] = t[1];
			}
		}
		System.gc();
		for (int i = 0; i < d.size(); i++) {
			for (int j = 0; j < e.size(); j++) {
				long[] t = getTime(d.get(i), e.get(j));
				res[i][j] = t[0];
				res[i+5][j] = t[1];
			}
		}
		System.out.println(googleRes(res));
	}
	
	private static final String APPLICATION_NAME = "Scheduler";
	private static final JsonFactory JSON_FACTORY = JacksonFactory.getDefaultInstance();
	private static final String REFRESH_TOKEN_PATH = "/refresh_token.txt";
	private static final String CREDENTIALS_FILE_PATH = "/credentials.json";
	private static String REFRESH_TOKEN;
	
	static {
		try {
			REFRESH_TOKEN = new BufferedReader(new InputStreamReader(GoogleFormService.class.getResourceAsStream(REFRESH_TOKEN_PATH))).readLine();
		} catch (IOException e) {
			e.printStackTrace();
		}
	}
	
	public static String googleRes(long[][] res) {
		try {
			final NetHttpTransport HTTP_TRANSPORT = GoogleNetHttpTransport.newTrustedTransport();
			Sheets sheetsService = new Sheets.Builder(HTTP_TRANSPORT, JSON_FACTORY, getCredentials(HTTP_TRANSPORT))
			.setApplicationName(APPLICATION_NAME)
			.build();
			
			Spreadsheet spreadSheet = sheetsService.spreadsheets()
			.get("11hYJlt4qiWWEKrj8AciEeMSbxH8ZSy8wgyEQWfrpNTk")
			.execute();
			List<List<Object>> list = new ArrayList<>();
			
			
			for (int i = 0; i < 20; i++) {
				List<Object> l = new ArrayList<>();
				for (int j = 0; j < 4; j++) {
					l.add(String.valueOf(res[i][j]));
				}
				list.add(l);
			}
			
			ValueRange body = new ValueRange().setValues(list);
			sheetsService.spreadsheets().values()
			.clear(spreadSheet.getSpreadsheetId(), "A1:H", new ClearValuesRequest())
			.execute();
			sheetsService.spreadsheets().values()
			.update(spreadSheet.getSpreadsheetId(), "A1:H", body)
			.setValueInputOption("RAW")
			.execute();
			return spreadSheet.getSpreadsheetUrl();
			
		} catch (Exception e) {
			System.out.println(e);
			return null;
		}
	}
	
	public static void main(String[] args) {
		long[][] res = new long[5][4];
		for (int i = 0; i < 5; i++) {
			for (int j = 0; j < 4; j++) {
				res[i][j] = i * j;
			}
		}
		googleRes(res);
	}
	
	public static Credential getCredentials(final NetHttpTransport HTTP_TRANSPORT) throws IOException {
		InputStream in = GoogleFormService.class.getResourceAsStream(CREDENTIALS_FILE_PATH);
		if (in == null) {
			throw new FileNotFoundException("Resource not found: " + CREDENTIALS_FILE_PATH);
		}
		GoogleClientSecrets clientSecrets = GoogleClientSecrets.load(JSON_FACTORY, new InputStreamReader(in));
		String clientId = clientSecrets.getDetails().getClientId();
		String clientSecret = clientSecrets.getDetails().getClientSecret();
		GoogleCredential credential = new GoogleCredential.Builder()
		.setTransport(HTTP_TRANSPORT)
		.setJsonFactory(JSON_FACTORY)
		.setClientSecrets(clientId, clientSecret)
		.build();
		credential.setAccessToken(getNewToken(REFRESH_TOKEN, clientId, clientSecret));
		credential.setRefreshToken(REFRESH_TOKEN);
		return credential;
	}
	
	public static String getNewToken(String refreshToken, String clientId, String clientSecret) throws
	IOException {
		ArrayList<String> scopes = new ArrayList<>();
		scopes.add(SheetsScopes.SPREADSHEETS);
		TokenResponse tokenResponse = new GoogleRefreshTokenRequest(new NetHttpTransport(), new JacksonFactory(),
		refreshToken, clientId, clientSecret).setScopes(scopes).setGrantType("refresh_token").execute();
		return tokenResponse.getAccessToken();
	}
}

\end{lstlisting}