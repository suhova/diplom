\chapter{Тестирование и апробация} \label{ch4}

В этой главе в параграфах \ref{ch4:sec1} и \ref{ch4:sec2} речь пойдёт о проводимом тестировании системы составления предварительного расписания сессии СПбПУ, а в параграфе \ref{ch4:sec3} о её апробации в период весенней сессии 20-21 учебного года.

\section{Функциональное тестирование} \label{ch4:sec1}
Для проверки того, что разработанная система выполняет все требуемые функции, было проведено ручное функциональное тестирование. Для этого были созданы несколько искусственных вариантов входных файлов, содержащих данные о типах аттестаций, учебных планах направлений и список аудиторий. С их помощью было проверены следующие функциональности:
\begin{itemize}
	\item при загрузке файла с аудиториями данные из него корректно сохраняются в базе данных и пользователю отображается новый список;
	\item при загрузке файла с учебным планом для конкретной специальности, в базе данных обновляется информация только по выбранной специальности;
	\item при загрузке файла с типами аттестаций информация из него сохраняется в базе данных, перетирая предыдущие данные о типах мероприятий, и отображается пользователю.
\end{itemize} 	

Также на тестовой google-таблице с данными о событиях с установленными датой, временем и аудиторией была проверена возможность учитывать эти данные, как константу при составлении остального расписания.

При выборе дат и времени в формах редактирования сессии, была проверена корректность обновления этих данных в базе данных и отображение их пользователю. Входные данные подбирались таким образом, чтобы было видно, что данные с сайта с расписанием учебных занятий \cite{ruz} учитываются при генерации нового расписания.

Для проектов, разрабатываемых длительное время командой программистов, вместо ручного тестирования следовало бы использовать автоматизированное, но для данного проекта разработка самих автотестов заняла бы больше времени, чем ручное их прохождение. Проводимые тесты воспроизводили сценарии работы с системой составления предварительного расписания сессии, и показали, что с точки зрения пользователя она выполняет все заявленные функции.

\section{Unit-тестирование} \label{ch4:sec2}

 Для отладки и финальной проверки правильности работы отдельных методов программы использовалось модульное тестирование. Насписание тестов позволяет на ранних этапах разработки отсекать ошибки, что повышает производительность и ускоряет процесс создания системы, так как корректность отдельных программных блоков проверяется сразу же, исключая ситуацию с переписыванием большого количества кода, вызванную чередой вовремя не замеченных багов.
 
 Для написания модульных тестов использовался java-фреймворк jUnit. Был создан тестовый класс SchedulerTest, код которого находится в приложении \ref{appendix-test}. Данный класс содержит методы, помеченные аннотацией @Test, в которых выполнение некоторого условия проверяется с мощью методов класса Assert. Класс Assert фреймворка jUnit имеет множество методов, удобных для сверки ожидаемого результата проверяемой функции с реальным. Так, assertArrayEquals позволяет сравнивать содержимое массивов, assertNull - равенство null, assertEquals - любое равенство и так далее. При невыполнении условий такие тесты выбрасывают исключение, что даёт разработчику сигнал о том, что изменения кода внесли ошибки в старый код, и прежде чем приступать к следующим, нужно доработать эти. Таким образом, с помощью модульного тестирования была проверена корректность работы алгоритма составленения предварительного расписания при разных наборах входных данных, что подтвердило его работоспособность и в случаях, когда расписание можно составить с учётом всех пожеланий, и когда приходится игнорировать некоторые пожелания, и когда расписание невозможно составить в принципе. 
 
 \section{Апробация} \label{ch4:sec3}
 Апробация системы составления предварительного расписания проводилась на весенней сессии 2020-2021 учебного года. 
 
 Первый этап апробации заключался в составлении расписания экзаменов для студентов 4-ых курсов направлений 02.03.03 и 09.03.03. Форма для сбора пожеланий преподавателей была выполнена с помощью сервиса Microsoft Forms, так как это позволяло автоматически определять адреса электронных почт и имена респондентов. На этом этапе были выявлены несколько её недостатков:

\begin{itemize}
	\item форма одинакова для всех преподавателей, и для того чтобы указать необходимые типы аудиторий преподавателю приходилось выбирать из огромного списка только те группы и дисциплины, у которых он проводит аттестации;
	\item имена ассистентов, введённые в поле респондентами не всегда соответствовали именам преподавателей с сайта расписания учебных занятий, что затруднило их маппинг;
	\item одному респонденту приходилось отправлять форму несколько раз для разных аттестаций;
	\item Microsoft Forms не имеет API для чтения собранных результатов;
	\item ручное создание формы заняло много времени.
\end{itemize} 	

Для решения этих проблем было принято решение делать формы с помощью сервиса Google Forms. Его преимуществом перед Microsoft является наличие API, позволяющего полностью автоматизировать как чтение результатов, так и создание самой формы. Теперь каждый преподаватель мог выбрать своё имя, а далее заполнять форму только по своим аттестациям. Также, в новой версии формы был минимизирован ручной ввод информации респондентами. Это решило проблему опечаток и различий введённых имён от имён из расписания. Новая версия формы не требовала повторной отправки, так как все необходимые данные преподаватель может ввести за одну итерацию заполнения формы.

Новый вариант google-формы был опробован при составлении предварительного расписания летней сессии студентов заочной формы обучения специальности Прикладная информатика. 
а также расписания их установочных занятий. Таким образом, были получены две таблицы с расписаними, сгенерованными системой, что продемонстрировало принципиальную применимость разработанной системы к составлению расписания любого мероприятия, занимающего непрерывное время в течение одного учебного дня. 

\section{Выводы} \label{ch4:conclusion}

Система составления предварительного расписания сессии была успешно применена в СПбПУ при составлении расписания сессии и установочных занятий студентов заочной формы обучения, чем доказала, что может быть использована для составления расписаний любых разовых мероприятий, занимающих непрерывное время в течение дня. Это делает её очень полезной для составления расписания дополнительной сессии, так как ей занимается напрямую кафедра без участия дирекции института, а значит, в этом случае предварительное расписание можно рассматривать как основанное.

Как и было заявлено в требованиях, система автоматизирует сбор сведений от профессорско-преподавательского состава, а так же сама получает и учитывает данные из официального расписания СПбПУ. Ручное редактирование расписания доступно в нескольких форматах - финальные ручные правки админимстратрором в сгенерированной google-таблице, либо внесение установленных аттестаций в таблицу учёта, и генерация нового расписписания на её основе.

Из потенциальных улучшений, которые в будущем можно добавить в систему, можно назвать:
\begin{itemize}
	\item создание более надёжного модуля аутентификации;
	\item рассылка преподавателям напоминаний о том, что они не заполнили форму сбора сведений;
	\item открытие и рассылка готово расписания преподавателям;
	\item создание модуля навигации между сгенерированными таблицами и форма сбора сведений.
\end{itemize} 	

Добавление этих и других модулей не будет вынуждать разработчика переписывать много существующего кода, так как всё это отдельные блоки, которые можно легко встраивать в текущую систему.