\chapter{Выбор средств реализации} \label{ch3}

% не рекомендуется использовать отдельную section <<введение>> после лета 2020 года
%\section{Введение} \label{ch3:intro}
В данной главе речь идёт о ПО, рассматривавшемся в качестве средств разработки ситемы составления расписания сессии для СПбПу. В параграфе \ref{ch3:sec1} приводится обоснования выбора языка программирования, в параграфе \ref{ch3:sec2} - фреймворка для работы с веб-составляющей системы, а в параграфе \ref{ch3:sec3} - рассказано о способе хранения и взяимодействия с данными.
	
\section{Выбор языка программирования} \label{ch3:sec1}
Так как система составления расписания сессии должна иметь веб-модуль, необходимо подобрать подхлдящий для этой цели язык программирования. Среди высокоуровневых языков, которые потенциально могли бы сделать процесс разработки более быстрым, а результат качественным рассматривались следующие претенденты: 
\begin{itemize}
\item  Java;
\item  JavaScript;
\item  C\#;	
\item  C++.	
\end{itemize}

\subsection{Java}
Java - один из самых пополуярных объектно-ориентированных языков программирования, что всягда является преимуществом, так как количество и качество документации, множество фреймворков и библиотек под любые нужды делает процесс разработки быстрее и эффективнее. Технология Garbage Collection, реализованная в Java оптимизирует управление памятью программ, что очень важно для работы с большим количеством входных данных, как в задаче составления расписания сессии. Также плюсом данного языка является его кроссплатформенность, а значит код на нём может быть запущен везде, где установлена Java Virtual Machine. 

\subsection{JavaScript}
JavaScript - язык программирования, который изначально использовался для разработки фронтенда и с помощью скриптов встраивал в HTML страницы исполняемый код. Сейчас JavaScript можно использовать и в качестве основного языка для написания бэкенда, так как c появлением среды Node.js программы на JavaScript можно запускать и на сервере. Преимуществом JavaScript всё ещё является его ориентированность на работу с веб страницами, а также его простота. Скрипты на этом языке поддерживаются всеми современными браузерами и не требуют установки дополнительного программноо обеспечения. В качестве языка для бэкенда недостатком служит отсутствие явной типизации, а так же разработку усложняет тот факт, что нет возможности узнать об ошибке на этапе компиляции, а значит о том, что где-то в коде появилось сложение строки с счислом, разработчик узнает только тогда, когда программа выполнится до этого места и не раньше. 

\subsection{C\#}
C\#, как и Java, является кроссплатформенным языком программирования, который разработан для создания приложений среды NET Framework. С точки зрения разработки C\# удобен обилием синтаксического сахара, за счёт которого упраздняются громоздкие конструкции, делающие написание кода более бастрым, а также повышающие его читаемость. Это в какой-то мере сказывается на производительности вычислений, но окупается удобством разработки. Помимо этого для C\# тоже существует множество фреймворков, что также предоставляет возможность переиспользовать уже готовые технологии.

\subsection{C++}
C++ - объектно-ориентрованный язык программирования с возможностью низкоуровневой работы с данными, что даёт возможность писать на нём даже микроконтроллеры. Но сложность синтаксиса и небогатая стандартная библиотека затрудняет высокоуровневую разработку. C++ достаточно популярен и имеет большое количество документации, но даже с ней не всегда легко правильно самостоятельно обеспечить грамотную работу с памятью. Ещё одним недостатком этого языка является зависимость от платформы и этот фактор перевешивает даже отличную производительность этого языка.

\subsection{Определение языка программирования}
Для реализации сервиса составления предварительного расписания сессии к языку программирования предъявлялись следующие требования:
\begin{itemize}
\item Наличие подробной докуменации;
\item Наличие фреймворков для упрощения работы с базой данных и веб-сервисом программы;
\item Кроссплатформенность;
\item Удобство и скорость разработки.
\end{itemize}

Язык Java удовлетворяет всем перечисленным критериям благодаря своей популяности среди программистов. Это существенно упрощает процесс разработки, так как для Java существует множество пособий, а способы устранения возникающих ошибок уже в большинстве собой описаны на различных интернет-ресурсах. Далее выбор фреймворков будет рассматриваться именно для Java, благо для данного языка их существует достаточно. Но стоит заметить, что язык JavaScript, хоть и не очень хорошо подходит для основоного языка в этом проекте, но будет использоваться для создания веб-интерфейса, так как Java не очень хорошо приспособлена для этой задачи.

\section{Выбор фреймворка для построения веб-приложения} \label{ch3:sec2}
\subsection{Spring}
Spring - Java-фреймворк, состоящий из множества компонентов таких как Inversion of Control для управления объектами, MVC для расширения Servlet API и прозрачного разделения между слоями модель, представление, контроллер и даже компонента доступа к базам данных \cite{spring}. Все эти и другие компоненты могут добавляться в проект независимо, что делает разработку более гибкой. Наличие этих модулей избавляет разработчика от низкоуровневой работы с запросами, налаживания взаимодействия с базами данных и прочих моментов имющих чисто технический характер и не имеющих связи с бизнес-задачей программы. Ещё одним преимуществом Spring  для разработчика является его популярность и качественная документация, чего часто не хватает молодым и мало известным проектам.

\subsection{Dropwizard}
Dropwizard - это фреймворк для создания веб-сервисов RESTful \cite{dropwizard}. Он по умолчанию поставляется с такими библиотеками, как Jetty server, Jackson, Metrics, Hibernate Validator и Guava. Dropwizard не имеет русскоязычной документации, что является недостатком. Так же у Dropwizard не так много модулей, как у Spring, но всё ещё просто расширяется с ипользованием файлов конфигураций.

\subsection{Vert.x}
Vert.x - это асинхронный, событийно ориентированный фреймворк, создатели которого во многом вдохновлялись фреймворком node.js \cite{vertx}. Он позволяет оптимизировать параллельную отправку и получение запросов, обрабатывая их большее количество с меньшими ресурсами, нежели фреймворки, основанные на блокирующем асинхронном вводе-выводе. Vert.x позволяет работать с действительно нагруженными системами, но не иммет такого же количества модулей для разных целей, как Dropwizard и тем более Spring.

\section{Определение фреймворка для построения веб-приложения} 
Для написания системы составления расписнаия сессии СПбПу из рассомтренных фреймворков был выбран именно Spring, потому что обладает всеми необходимыми для данной задачи модулями. Огромное количество библиотек, включённых в Spring может помочь при дальнейшем расширении сервиса без переписывания его на другой фреймворк. Также при поддержании работы системы другими разработчиками не возникнет проблемы с изучением Spring, потому что для него существует множество справочных материалов в том числе на русском языке.

\section{Хранение данных} \label{ch3:sec3}
Spring Framework имеет поддержку ORM Hibernate \cite{hibernate}. ORM или Object-Relational Mapping - это технология, позволяющия сопоставлять объекты объекно-ориентрованной модели, на которой зиждится Java, с сущностями базы данных. Таким образом, не нужно вручную создавать таблицы с полями, соответствующими структуре Java - объекта, ведь можно просто добавить к нему аннотацию, а Hibernate сделает всё автоматически, а самое главное автоматически преобразует полученные результаты select-запроса в объект, что не всегда просто делать самостоятельно. Также, Hibernate имеет предопределённые запросы к базе данных, из-за чего можно не прописывать монотонные sql-запросы на поиск и удаления элемента по id или добавления его в таблицу.

Hibernate поддерживает диалекты и MySQL, и Oracle, и Microsoft SQL Server, и PostgreSQL, поэтому в качестве СУБД была выбрана PostgreSQL, так как она полностью бесплатна и проста в установке на любой платформе.

\section{Выводы} \label{ch3:conclusion}
Таким образом, в качестве основного языка программирования был выбран Java, который во фронтенд части дополняется JavaScript, более приспособленным для реализации интерфейсов. Среди фреймворков для работы с веб-приложениями для Java выбор пал на Spring Framework, имеющий множество разным модулей, в том числе Hibernate, с помощью которого осуществляется взаимодействие с базой данной PostgreSQL.

\begin{comment}
Текст выводов по главе \thechapter.
\section{Название параграфа} \label{ch2:sec-abbr} %название по-русски

Название параграфа оформляется с помощью команды \verb|\section{...}|, название главы --- \verb|\chapter{...}|. 
\end{comment}